\documentclass[10pt]{letter}
\usepackage{UPS_letterhead,xcolor,mhchem,ragged2e,hyperref}
\usepackage{physics,siunitx}
\newcommand{\alert}[1]{\textcolor{red}{#1}}
\definecolor{darkgreen}{HTML}{009900}

\newcommand{\yann}{\textcolor{green}}
\newcommand{\Ndet}{N_\text{det}}


\begin{document}

\begin{letter}%
{To the Editors of the \textit{Journal of Chemical Theory and Computation},}

\opening{Dear Editors,}

\justifying
Please find attached a revised version of the manuscript entitled
\begin{quote}
	\textit{``Seniority and Hierarchy Configuration Interaction for Radicals and Excited States''}
\end{quote}

We thank the reviewers for their support to publication of the present manuscript.
Our detailed responses to their comments can be found below.
For convenience, changes of the revised manuscript are highlighted in red in a separate file.

We look forward to hearing from you.

\closing{Sincerely, the authors.}

\newpage

%%% REVIEWER 1 %%%
\noindent \textbf{\large Authors' answer to Reviewer \#1}

{In this manuscript, Loos and coworkers report the extension of the hCI method (J. Phys. Chem. Lett. 2022, 13, 4342) towards calculating the open-shell and excited-state chemical species.
This method can be seen as a partitioning of the FCI space that combines the excitation-based CI (eCI, which is a conventional truncated CI) and the seniority-based CI (sCI).
The authors report various benchmark data for radicals (\ce{OH}, \ce{CN}, vinyl, \ce{H7}) and excited states (the molecules at the geometries from the QUEST database).
The authors also introduced the second-order Epstein-Nesbet perturbative corrections (EN2) in the hCI calculations,
and the impact of such correction is described for the closed-shell systems as well.
Overall, this work reasonably demonstrates applications of novel electronic-structure methods for diverse systems.
The paper is interesting, well-written, and straightforward to follow, and I think this paper deserves publication in J. Chem. Theory Comput. after minor improvements.
I have only a couple of brief comments below:
}

\alert{
We thank the reviwer for the supportive comments.
}


{This work presents the partitioning of the Hilbert space. Although the authors present the number
of determinants in the appendix, can the authors also provide the number of the FCI determinants
so that readers can easily evaluate the efficiency of the xCI calculations?
}

\alert{
The number of determinants in FCI and in each approximate CI model is now presented in Tables S1 and S2 of the Supporting Information.
}

{Figures 3 and 4: The data is shown with respect to the number of determinants, which makes the
reader to evaluate the different methods efficiently. However, for readability, I think it will also be
helpful to provide n in eCIn, sCIn, and hCIn in these figures.
}

\alert{
We have experimented including the values of $n$ to the figures, as suggested by the reviewer.
However, we consider that they become somewhat clutered and less readable, and therefore we prefer to leave them unchanged.
With the intention of helping the reader, we have been supplemented the following sentence to the captions of Figures 3 (4):
``Each point denotes one CI model, according to the sequences: HF, hCI1, hCI1.5, hCI2, etc. (green); HF, CIS, CISD, etc. (red); and sCI0, sCI2, and sCI4 (sCI1, sCI3, sCI5) (blue).''
}

{In sections IV-C and IV-D: The EN2 corrections are not discussed at all in the excited states, making
it unnatural compared to the presentation of the radical systems. While it might be challenging to
perform the additional calculations for all the systems, I think it will be helpful for some sample
systems to investigate the impact of EN2 correction in hCI and DhCI calculations (maybe in the
same fashion presented in Figures 3 and 4).
}

\alert{
We agree with the reviewer's suggestion that including the EN2 correction for the excited state calculations is a natural and important addition to the manuscript.
We have performed additional calculations for the full set of states with the different CI models surveyed here (except for the more computationally demanding higher-order models).
These new results were included in Tables II, III, and in the Supporting Information, while a new subsection (IV.E), starting on page x, is devoted to their discussion.
Additions to the Conclusion (page x) and the abstract have also been made.
}

\clearpage

%%% REVIEWER 2 %%%
\noindent \textbf{\large Authors' answer to Reviewer \#2}

{The authors present a generalization of the hierarchy configuration interaction (hCI) method to an arbitrary reference state, which allows them to extend their previously developed hCI model to open-shell systems and general excited-state electronic structures. They test the performance of their proposed extension against various test sets containing (small) molecules for which reliable reference data is available. Their numerical analysis and discussion are in-depth, while their methodology (the theory part) is comprehensive. Their numerical study allows them to obtain reliable energies and spectroscopic properties for their more elaborate hCI models. This manuscript is of potential interest to the readership of the Journal of Chemical Theory and Computation. However, a few remarks need to be addressed before publication.
}

\alert{
We thank the reviwer for the positive appreciation of our work.
}

{1) The sentence ``[...] whose number $\Ndet$ share the same scaling with the number of basis functions $N$ [...]'' is rather difficult to understand. The authors need to either rephrase or elaborate more.
}

\alert{
This sentence has been rephrased and now reads, on page x,
``The number of determinants $\Ndet$ within a given class scales polynomially with the number of basis functions $N$, with the exponent depending on the specific class.
For instance, for the class of doubly-excited determinants with no unpaired electrons ($e=2$ and $s=0$), $\Ndet = \order*{N^2}$.
hCI was defined such that, at a given hierarchy $h$, all classes of determinants presenting a scaling of $\Ndet = \order*{N^{2h}}$ or less are accounted for.''
}

{2) the authors need to include more information concerning the fitting procedure in the main manuscript for reproducibility reasons. In the SI, they state that they used a Morse function. Why did the authors use a Morse function and not, for instance, a generalized Morse function, etc... Furthermore, it is unclear how they fitted their vibrational frequencies.
}

\alert{
The reviewer's questions are addressed on page x of the revised Supporting Information, reading
``By fitting the computed potential energy curves with a Morse potential $V(r) = D [1 - e^{-a(r-r_e)}]^2 + C $,
we obtained equilibrium geometries $r_e$, force constants $2Da^2$, and harmonic vibrational frequencies $\left( Da^2/2\pi^2\mu \right)^{1/2}$,
where $\mu$ is the reduced mass.
Since the fitting is restricted to the equilibrium region, employing more elaborate functions would be inconsequential, such that the adopted Morse potential is well justified.
The fitting was performed with the Marquardt-Levenberg algorithm implemented in Gnuplot.''
%Since the fitting is restricted to the equilibrium region (the ranges are given in the Supporting Information),
%using a more elaborate functional form would only produce minimal changes to the obtained fitted values. A Morse potential is thus enough for our purposes.
%Moreover, any residual bias from the fitting function would be inconsequential to the goal of comparing to the (equally fitted) FCI values.
We prefer to keep these details in the Supporting Information rather than in the main text.
}

{3) Some of the PES displayed in the SI are not smooth and rather kinky. How does the lack of smoothness affect the fitting procedure and, hence, the vibrational frequencies?
}

\alert{
\ce{CN} is the only system for which some of the potential energy curves are not smooth around the equilibrium geometry.
This is mentioned on page x of the revised manuscript, reading
``For the \ce{CN} radical, CIS, sCI1, and sCI3 produce crossings between ground and excited states around equilibrium, hence non-smooth adiabatic PECs
and somewhat less reliable equilibrium properties in these specific cases.''
}

\clearpage

{4) The authors need to mention (or draw) the molecules of their excited state study. Not everyone might be familiar with the QUEST database. It would be insightful to mention the character of the excited states. I assume these are all singly excited states (no double excitation character?). Can the authors provide additional information regarding their nature (local excitation, charge transfer, Rydberg, etc.)? Specifically, what kind of excited states can be described well or less well with hCI methods?
}

\alert{
The set of systems we considered is now presented in Table II, whreas on page x we describe the character of the excited states, reading
``Our set includes mostly small systems, ranging from \ce{BH} to glyoxal, displaying a mix of valence and Rydberg singly-excited states,
and 2 doubly-excited states (glyoxal and nitroxyl). It does not include large molecules nor charge transfer states.''
The details on each specific excited state can be found in The Supporting Information.
The revised manuscript now mentions the performance of the hCI models for different types of excited state, on page x:
``$\Delta$hCI2 is more accurate for singlets than for triplets (MAEs of \SI{0.16}{\eV} and \SI{0.25}{\eV}),
whereas Rydberg states are better described than valence states (MAEs of \SI{0.14}{\eV} and \SI{0.24}{\eV}).
The same trends are found for the low-order models ($\Delta$hCI1 and $\Delta$hCI1.5), and also for $\Delta$CISD, which slightly outperforms $\Delta$hCI2 for each type of excitation.
The specific MAEs can be found in the Supporting Information.''
This aspect is also discussed for the hCI+EN2 models in the new subsection (IV.E).
}

{5) For reasons of transparency, the authors should define MSE, MAE, RMSE, and SDE using equations, even if it might be obvious.
}

\alert{
The statistical measures are now defined on page x of the Supporting Information, which are refered to on page x of the main text, reading
``For completeness, the defition of these statistical measures can be found in the Supporting Information.''
}

{6) The authors keep mentioning ``reference theoretical values.'' Please mention what kind of reference values were used.
}

\alert{
This information is provided in the Supporting Information, which is stated on page x of the revised manuscript, reading
``For each excited state considered here, the reference value and the corresponding method to obtain it (of high-order coupled cluster or extrapolated FCI quality) can be found in the Supporting Information.''
}

{7) The authors look at several small-sized molecules for which FCI can still be determined. There is nothing wrong with that. However, I am missing numerical studies on slightly larger systems, which are known to be challenging in computational chemistry. I recommend adding some ``larger'' molecules, which feature open- and closed-shell ground and excited states. Examples are cyclobutadiene (its automatization problem) and o-, m-, and p-benzynes. These systems are notoriously difficult to study and extensively discussed in the literature. Including just those four molecules would add more value to the paper and allow a direct comparison to other state-of-the-art computational methods.
}

\alert{
This is a pertinent remark.
From the examples suggested by the reviewer, we decided to investigate the automerization problem of cyclobutadiene.
Our results can be found in a new paragraph on page x, starting as:
``We further explored hCI to describe the automerization barrier in the ground state of cyclobutadiene...''
On the other hand, the case of benzynes is left as a perspective for the future, otherwise the manuscript would be excessively lengthy in our opinion.
We notice that extensive CI+EN2 additional calculations have been carried out and a corresponding new subsection (IV.E) has been added to the manuscript, based on a suggestion from Reviewer \#1.
We further highlight the point raised by Reviewer \#2 in the Conclusion, reading
``hCI also produced reasonable energies for the automerization barrier of cyclobutadiene.
It remains to be seen how these models perform for other strongly correlated systems, larger molecules, and charge transfer excited states.''
}

\end{letter}
\end{document}
