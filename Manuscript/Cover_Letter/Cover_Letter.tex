\documentclass[10pt]{letter}
\usepackage{UPS_letterhead,color,mhchem,mathpazo,ragged2e}
\newcommand{\alert}[1]{\textcolor{red}{#1}}

\usepackage[
        colorlinks=true,
    citecolor=blue,
    breaklinks=true
        ]{hyperref}
\urlstyle{same}

\begin{document}

\begin{letter}%
{To the Editors of the The Journal of Chemical Theory and Computation}

\opening{Dear Editors,}

\justifying
Please find enclosed our manuscript entitled \textit{``Seniority and hierarchy configuration interaction for radicals and excited states''}, 
which we would like you to consider as a Regular Article to the \textit{Journal of Chemical Theory and Computation}.
This contribution has never been submitted in total nor in parts to any other journal, and has been seen and approved by all authors.

Describing static and dynamic correlation on an equal footing is an ever-present goal in the development of novel electronic structure methods.
With this motivation in mind, we have recently proposed hierarchy configuration interaction (hCI)
[\href{https://doi.org/10.1021/acs.jpclett.2c00730}{\textit{J.~Phys.~Chem.~Lett.}~\textbf{2022}, \textit{13}, 4342}],
a new route that presented superior performance to the traditional excitation-based and the more recent seniority-based alternatives.
Despite the promising initial results, hCI had only been defined for ground-state closed-shell systems.

In the present contribution, we report a general formulation of hCI, significantly expanding its domain of applicability.
By tackling challenging and diverse chemical problems, including the dissociation of radical species and the excited states of closed- and open-shell systems,
hCI is extensively gauged against other configuration interaction (CI) alternatives.
The present findings establish hCI as an appealing method for recovering static and dynamic correlation.
They further encourage the development of hierarchy-based coupled-cluster methods, along with corresponding equation-of-motion formulations for excited states.

This contribution also concerns seniority-based CI for excited states.
Despite the growing interest seniority-based methods have attracted over the past decade,
here we demonstrate its poor performance in the calculation of excitation energies for closed-shell systems.
The results are more encouraging for excitations of open-shell systems,
which is relevant for the pursuit of seniority-based coupled cluster methods.

We suggest Paul Johnson, Gianluca Levi, Hannes Jonsson, Stijn de Baerdemacker, Eric Neuscamman, and Katharina Boguslawski as potential referees.
We look forward to hearing from you soon.

\closing{Sincerely,}

\end{letter}
\end{document}
