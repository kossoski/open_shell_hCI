\documentclass[aip,jcp,reprint,noshowkeys,superscriptaddress]{revtex4-1}
\usepackage{graphicx,dcolumn,bm,xcolor,microtype,multirow,amscd,amsmath,amssymb,amsfonts,physics,wrapfig,txfonts,siunitx}
\usepackage[version=4]{mhchem}
%\usepackage{natbib}
\bibliographystyle{achemso}

\newcommand{\ie}{\textit{i.e.}}
\newcommand{\eg}{\textit{e.g.}}
\newcommand{\alert}[1]{\textcolor{black}{#1}}
\usepackage[normalem]{ulem}
\newcommand{\fk}[1]{\textcolor{blue}{#1}}
\newcommand{\titou}[1]{\textcolor{red}{#1}}
\newcommand{\trashPFL}[1]{\textcolor{red}{\sout{#1}}}
\newcommand{\PFL}[1]{\titou{(\underline{\bf PFL}: #1)}}
\newcommand{\toto}[1]{\textcolor{green}{#1}}
\newcommand{\trashAS}[1]{\textcolor{green}{\sout{#1}}}
\newcommand{\AS}[1]{\toto{(\underline{\bf AS}: #1)}}
\newcommand{\ant}[1]{\textcolor{orange}{#1}}
\newcommand{\SupInf}{\textcolor{blue}{Supporting Information}}

\newcommand{\mc}{\multicolumn}
\newcommand{\fnm}{\footnotemark}
\newcommand{\fnt}{\footnotetext}
\newcommand{\tabc}[1]{\multicolumn{1}{c}{#1}}
\newcommand{\QP}{\textsc{quantum package}}

\newcommand{\EHF}{E_\text{HF}}
\newcommand{\EDOCI}{E_\text{DOCI}}
\newcommand{\EFCI}{E_\text{FCI}}
\newcommand{\Ndet}{N_\text{det}}
\newcommand{\Nbas}{N}

\usepackage[
	colorlinks=true,
    citecolor=blue,
    breaklinks=true
	]{hyperref}
\urlstyle{same}

\begin{document}

\newcommand{\LCPQ}{Laboratoire de Chimie et Physique Quantiques (UMR 5626), Universit\'e de Toulouse, CNRS, UPS, France}

\title{Seniority and hierarchy configuration interaction for radicals and excited states}

\author{F\'abris Kossoski}
\email{fkossoski@irsamc.ups-tlse.fr}
\affiliation{\LCPQ}
\author{Pierre-Fran\c{c}ois Loos}
\email{loos@irsamc.ups-tlse.fr}
\affiliation{\LCPQ}

% Abstract
\begin{abstract}
{\bf Abstract:} 
%\bigskip
%\begin{center}
%        \boxed{\includegraphics[keepaspectratio,width=2in]{TOC}}
%\end{center}
%\bigskip
\end{abstract}

% Title
\maketitle


%%%%%%%%%%%%%%%%%%%%%%%%%%%%%%%%%%%%%%%%%%%%%%%%%%
\section{Introduction}
\label{sec:intro}
%%%%%%%%%%%%%%%%%%%%%%%%%%%%%%%%%%%%%%%%%%%%%%%%%%


Configuration interaction (CI) offers a systematic way to solve the many-body electronic structure problem. \cite{SzaboBook,Helgakerbook}
By including progressively more determinants in the Hilbert space, the wave function becomes increasingly closer to the exact one, and so does the electronic energy.
In full CI (FCI), all determinants are accounted for and the problem is solved exactly (for a given basis set).
In practice, however, one resorts to approximate CI methods, where only the determinants that satisfy a given criterion are included in the truncated Hilbert space.
For small systems, FCI is still attainable and one can thus gauge the performance of approximate CI methods.
Excitation-based CI
Seniority based CI (sCI) \cite{Bytautas_2011,Allen_1962,Smith_1965,Veillard_1967}
sCI works well to describe molecular dissociation \cite{Bytautas_2015,Alcoba_2014,Alcoba_2014b}
Limited studies for radicals and excited states

We have recently introduced hierarchy CI (hCI), \cite{Kossoski_2022}
where the Hilbert space is partitioned according to a hierarchy parameter $h$ that combines the excitation degree $e$ and the seniority number $s$.
For different properties and molecular systems, hCI showed an improved convergence to the FCI results when compared to the traditional excitation-based CI. \cite{Kossoski_2022}
However, hCI has only been defined for a closed-shell reference, thus systems with an even number of electrons.
A generalization of the method for radicals has not yet been proposed.

The higher-lying CI roots give access to excited states.
Whereas the performance of excitation-based CI for excited states is well stablished, \cite{} this remains an open question for sCI methods,
despite the growing number of methods targeting excited states that exploit the concept of seniority number. \cite{} 
Similarly, the favourable results of hCI when compared to excitation-based CI for ground states raises the question on how well it can describe excitation energies.

When targeting excited states with CI methods, one can employ the ground-state HF orbitals or instead optimize the orbitals for each state of interest, in a so-called state-specific approach.
Promising results for state-specific excitation-based CI.
\cite{Kossoski_2023}
Here we introduce analogous state-specific seniority-based CI ($\Delta$sCI) and hierarchy-based CI ($\Delta$hCI).
How do these methods perform for excited states, from both closed-shell and open-shell systems?

Here we address the knowledge gaps outlined above by setting four goals.
Our first goal is to generalize hCI \cite{Kossoski_2022} for the case of an open-shell reference determinant, thus extending its domain to radical species.
The second is to assess the performance of this method for ground state radicals, similarly to what has been done for closed-shell systems. \cite{Kossoski_2022}
%Seniority hCI
The third goal is to assess the improvement from perturbative corrections to ground-state radicals and closed-shell systems, for both excitation-based and hierarchy-based CI methods.
Finally, we introduce and gauge seniority-based and hierarchy-based CI methods for excited states, along both the standard ground-state based route and the state-specific route. \cite{Kossoski_2023}


%%%%%%%%%%%%%%%%%%%%%%%%%%%%%%%%%
\section{Hierarchy configuration interaction}
\label{sec:hCI}
%%%%%%%%%%%%%%%%%%%%%%%%%%%%%%%%

%For an arbitrary reference Slater determinant, with seniority $s_0$, the hierarchy $h$ associated to a another determinant
With respect to a given reference Slater determinant, the hierarchy $h$ of a target determinant is defined as
\begin{equation}
        \label{eq:h}
        %h = \frac{e+\Delta s/2}{2},
	h = \frac{e+ (s-s_0)/2}{2},
\end{equation}
where $s$ and $s_0$ denote the seniority of target and reference determinants, respectively, and $e$ represents the excitation degree that connect the two determinants.
This definition renders a degree of dissimilarity between two determinants (reference and target),
which accounts for differences in orbital occupation (through the excitation degree $e$)
and differences in the number of unpaired electrons (through the term $(s-s_0)/2$).
The latter always assumes an integer value (for both even and odd numbers of electrons), such that $h$ becomes an integer or half-integer.
Also, Eq.~\eqref{eq:h} simplifies to the previous definition \cite{Kossoski_2022} for the case of a closed-shell reference determinant, when $s_0 = 0$.

From here on, method names carrying the $\Delta$ symbol denote a state-specific approach,
whereas those without the $\Delta$ mean that ground-state HF orbitals were employed.


%%%%%%%%%%%%%%%%%%%%%%%%%%%%%%%%%
\section{Computational details}
\label{sec:compdet}
%%%%%%%%%%%%%%%%%%%%%%%%%%%%%%%%

To gauge the performance of hCI against excitation-based and seniority-based CI for radical systems,
we have calculated the potential energy curves (PECs) for the dissociation of four radical systems:
\ce{OH}, \ce{CN}, vinyl (\ce{C2H3}), and \ce{H7}.
%which display a variable number of bond breaking.
The equilibrium geometry of vinyl was taken from Ref.~\onlinecite{Loos_2020},
%also reproduced in the \SupInf,
and the PECs were computed along the \ce{C=C} double bond breaking coordinate, with the remaining internal coordinates kept frozen.
For \ce{H7}, we considered equally spaced and linearly arranged hydrogen atoms, and the PECs were computed along the symmetric dissociation coordinate.
From the PECs, we performed a similar analysis as in our paper on hCI for closed-shell systems. \cite{Kossoski_2022}
Namely, for the different CI levels considered here, 
we evaluated the convergence of the non-parallelity error (NPE), the distance error, the vibrational frequencies, and the equilibrium geometries, as functions of $\Ndet$.
The NPE (distance error) of a given level of theory is defined as the maximum minus (plus) the minimum differences between its corresponding PEC and the FCI PEC.
Details about how the vibrational frequencies and equilibrium geometries were obtained from the calculated PECs can be found in the \SupInf.

To evaluate how the different levels of CI introduced here perform for excited states,
we calculated vertical excitation energies for 60 electronic states, comprising 26 different molecular systems, with geometries extracted from the QUEST database. \cite{Veril_2021}
% 18 closed-shells, 8 radicals
We employed the aug-cc-pVDZ basis set for systems having up to three non-hydrogen atoms and the 6-31+G(d) basis set for the larger ones.
We performed calculations following both the standard ground-state based CI route and the state-specific CI route. \cite{Kossoski_2023}
For the latter, we employed the state-specific orbitals obtained in Ref.~\onlinecite{Kossoski_2023}.
The computed excitation energies were compared against the reference values from the QUEST database. \cite{Veril_2021}
The full set of excited states and calculated excitation energies, for the various methods considered here, are provided in the {\SupInf}.
%For completeness, we also reproduce the geometries in the \SupInf.

The hCI methods introduced here were implemented in {\QP} \cite{Garniron_2019} through a straightforward modification of the
\textit{configuration interaction using a perturbative selection made iteratively} (CIPSI) algorithm. \cite{Huron_1973,Giner_2013,Giner_2015,Garniron_2018}
By allowing only for the determinants that are connected with the reference determinant(s) up to a given maximum hierarchy $h$,
the CIPSI algorithm is restricted to the truncated Hilbert space defined by the reference and the value of $h$.
{\QP} \cite{Garniron_2019} was employed to perform all the CI calculations presented here.
In a given calculation, the energies are considered to be converged when the (largest) Epstein-Nesbet second-order perturbation correction computed in the truncated Hilbert space 
lies below \SI{0.01}{\milli\hartree}. \cite{Garniron_2018}
%by spanning the most important regions of the truncated Hilbert space
This selected CI procedure requires considerably fewer determinants than the total number of determinants in the truncated Hilbert space,
while delievering fairly converged absolute and excitation energies.
The ground- and excited-state CI energies are obtained with the Davidson iterative algorithm. \cite{Davidson_1975}

state-specific methods
$\Delta$sCI2 
$\Delta$hCI1
%restricted open-shell HF orbitals


%%%%%%%%%%%%%%%%%%%%%%%%%%%%%%%%
\section{Results and discussion}
\label{sec:res}
%%%%%%%%%%%%%%%%%%%%%%%%%%%%%%%%


%%%%%%%%%%%%%%%%%%%%%%%%%%%%%%%%
\subsection{hCI for radicals}
\label{sec:res_A}
%%%%%%%%%%%%%%%%%%%%%%%%%%%%%%%%

% redefine the distance errors because of negative values?
% vinyl missing point at 5.2 for s3i+pt2 and at 2.0 for s5+pt2
% CN correct dissociating part for h4 and h3.5
% H8 missing points for Q+pt2
% H8 missing 9.0 for T+pt2

The full set of PECs for the open-shell systems (\ce{OH}, \ce{CN}, \ce{H7}, and vinyl) are presented in the {\SupInf},
along with the NPEs, distance errors, equilibrium geometries and vibrational frequencies, plotted as a function of $\Ndet$ for the different CI methods.
The present results for open-shell systems are discussed in detail in the following,
and display similar trends to the ones previously reported for closed-shell systems. \cite{Kossoski_2022}
hCI typically improves the convergence of the different properties when compared to excitation-based CI.
Concerning the NPEs, such improvement is clear for \ce{OH} and \ce{CN} (involving a single bond breaking), less so for vinyl (double bond breaking),
whereas for \ce{H7} (multiple bond breaking), hCI and excitation-based CI perform similarly.
This behavior had also been observed for dissociation of closed-shell systems. \cite{Kossoski_2022}
The same trend is observed for the distance error, although the improvement from excitation-based CI to hCI is less marked than for the NPEs.
hCI further leads to a faster convergence of the equilibrium geometry of \ce{CN}, while comparable results are found for the geometries of the other systems.
Meanwhile, a clear improvement from excitation-based CI to hCI is observed for the vibrational frequencies, except for \ce{H7}, where no difference is found.

%HF orbitals
%orbital optimization?

% rPT2 vs PT2 

We further assesssed the effect of the EN2 perturbative correction for both the open-shell systems studied here 
and the closed-shell systems (\ce{HF}, \ce{F2}, \ce{N2}, ethylene, \ce{H4}, and \ce{H8}) addressed in our first report on hCI. \cite{Kossoski_2022}
The full set of PECSs is shown in the {\SupInf}.
At stretched geometries, the EN2 correction can lead to kinks and large jumps in the PECSs.
Such problems were encountered at some CI methods for \ce{CN}, vinyl, \ce{H8}, and \ce{H7}, and to a less extent, for \ce{N2} (only for hCI1).
For the remaining five systems, the EN2 correction did not produce problematic PECs.

Around the equilibrium geometry, the EN2 correction produced well-behaved PECs at all levels of CI and for all systems, 
except for \ce{CN}, which is related to its different states around equilibrium.
%
xe
\ce{OH} massive improvement
\ce{CN} \ce{H4} \ce{N2} \ce{F2} \ce{HF} big improvement
\ce{H7} \ce{H8} vinyl ethylene some improvement
%
freq
\ce{OH} massive improvement
\ce{HF} \ce{H8} vinyl ethylene some improvement
\ce{CN} \ce{F2} \ce{N2} \ce{H4} \ce{H7} big improvement


%%%%%%%%%%%%%%%%%%%%%%%%%%%%%%%%
\subsection{hCI for excited states}
\label{sec:res_B}
%%%%%%%%%%%%%%%%%%%%%%%%%%%%%%%%

In Tab.~\ref{tab:1} we present the statistical errors of the different methods, organized by their computational scaling.
We found that $\Delta$hCI2 and $\Delta$CISD display comparable MAEs, despite the somewhat more favourable MSE and RMSE of the latter.
$\Delta$hCI2 still tends to underestimate the excitation energies, though to a much lesser extent than observed for the lower-order hCI methods.

\begin{table}[ht!]
\caption{Mean Signed Error (MSE), Mean Absolute Error (MAE), and Root-Mean Square Error (RMSE) in Units of eV, with Respect to Reference Theoretical Values,
for the Set of 60 Excitation Energies Listed in the {\SupInf}.
}
\label{tab:1}
\begin{ruledtabular}
\begin{tabular}{lddd}
method            &     \mc{1}{c}{MSE} & \mc{1}{c}{MAE} & \mc{1}{c}{RMSE} \\
\hline
$\Delta$CSF       & -0.54 & 0.70 & 0.86 \\
CIS               & +0.13 & 0.59 & 0.77 \\
\hline
hCI1              & +1.11 & 1.14 & 1.39 \\
$\Delta$hCI1      & -0.90 & 1.17 & 1.40 \\
\hline
hCI1.5            & +1.88 & 1.88 & 2.20 \\
$\Delta$hCI1.5    & -1.99 & 2.12 & 2.58 \\
\hline
$\Delta$CISD      & -0.08 & 0.17 & 0.22 \\
$\Delta$hCI2      & -0.15 & 0.21 & 0.30 \\
\hline
sCI2              &  0.96 & 1.06 & 1.28 \\
$\Delta$sCI2      &  0.39 & 0.60 & 0.73 \\
%\hline
%CIS(D)            & -0.03 & 0.35 & 0.40 \\
%CC2               & -0.05 & 0.32 & 0.37 \\
%CCSD              & +0.01 & 0.08 & 0.09 \\
%CC3               & +0.01 & 0.03 & 0.06 \\
%CCSDT             & -0.01 & 0.02 & 0.02 \\
\end{tabular}
\end{ruledtabular}
\end{table}


Despite the rather poor performance of lower-order hCI methods for excitation energies,
we further investigated whether higher-order versions could be competitive against the traditional excitation-based counterparts.
For that, we performed calculations for a subset of 14 excited states of small molecular systems (\ce{H2O}, \ce{BF}, \ce{HCF}, \ce{HCl}, and \ce{N2}),
with corresponding statistical errors shown in Tab.~\ref{tab:2}.
Even though this is a small set, it should be enough to obtain the main trends on the statistics of the different methods.


\begin{table}[ht!]
\caption{Mean Signed Error (MSE), Mean Absolute Error (MAE), and Root-Mean Square Error (RMSE) in Units of eV, with Respect to Reference Theoretical Values,
for the Set of 14 Excitation Energies Listed in the {\SupInf}.
}
\label{tab:2}
\begin{ruledtabular}
\begin{tabular}{lddd}
method            &     \mc{1}{c}{MSE} & \mc{1}{c}{MAE} & \mc{1}{c}{RMSE} \\
\hline
%CIS               & +0.16 & 0.82 & 0.96  \\
%$\Delta$CSF       & -0.84 & 0.91 & 1.03  \\
%\hline
CISD              & +4.09 & 4.09 & 4.18  \\
hCI2              & +2.89 & 3.00 & 3.28  \\
$\Delta$CISD      & -0.13 & 0.17 & 0.21  \\
$\Delta$hCI2      & -0.22 & 0.22 & 0.27  \\
\hline
hCI2.5            & +1.93 & 1.93 & 2.05  \\
$\Delta$hCI2.5    & -0.27 & 0.27 & 0.30  \\
\hline
CISDT             & +0.09 & 0.18 & 0.20  \\
hCI3              & +0.18 & 0.18 & 0.19  \\
$\Delta$CISDT     & -0.19 & 0.19 & 0.23  \\
$\Delta$hCI3      & -0.21 & 0.21 & 0.24  \\
\hline
$\Delta$hCI3.5    & -0.08 & 0.08 & 0.09  \\
\hline
CISDTQ            & +0.14 & 0.14 & 0.16  \\
$\Delta$CISDTQ    & -0.02 & 0.02 & 0.02  \\
\end{tabular}
\end{ruledtabular}
\end{table}

%We found that the errors of both ground-state based hCI and state-specific $\Delta$hCI to systematically decrease when going toward higher orders of $h$, which would be expected.
%Unfortunately, the pace of decrease is too slow to render these approaches very useful.
%In line with what has been found for excitation-based CI \cite{Kossoski_2023}, 

The case of $\Delta$hCI2 was discussed above for the larger set of excitation energies.
Here we further notice that hCI2 was found to be somewhat more accurate than CISD, even though the errors are still too large.
However, this improvement does not reflect on the state specific route (compare $\Delta$hCI2 and $\Delta$CISD in Tabs.~\ref{tab:1} and ~\ref{tab:2}).
Moving one step further, even though the state-specific route ($\Delta$hCI2.5) is far more accurate than the ground-state based route (hCI2.5),
it is actually less accurate then $\Delta$hCI2.
An improved performance is found for the methods displaying an $\order*{N^8}$ computational scaling.
The four methods, hCI3, $\Delta$hCI3, CISDT, and $\Delta$CISDT, display quite similar MAEs and RMSEs.
Therefore, there is no advantage by moving from excitation-based to hierarchy-based CI at this order, just as discussed above for the $\Delta$hCI2 and $\Delta$CISD.
Similarly, the $\Delta$hCI3 does not improve with respect to hCI3, as also observed for $\Delta$CISDT and CISDT. \cite{Kossoski_2023}
We notice, however, the negative MSEs for the state-specific methods, in contrast to the positives values obtained for the ground-state based route.
At the next order, $\Delta$hCI3.5 has significantly smaller errors, but at a costly $\order*{N^9}$ scaling.



%%%%%%%%%%%%%%%%%%%%%%%%%%%%%%%%
\subsection{Seniority CI for excited states}
\label{sec:res_C}
%%%%%%%%%%%%%%%%%%%%%%%%%%%%%%%%

We refer back to Tab.~\ref{tab:1} to discuss the performance of seniority-based CI for excited states.
sCI2 delivers quite poor results (MAE of \SI{1.06}{eV}), whereas $\Delta$sCI2 reduces the statistical errors (MAE of \SI{0.60}{eV}).
This improvement, however, is not as expressive as typically observed for excitataion-based and hierarchy-based methods.
(Compare, for instance, with the cases of hCI2.5 and $\Delta$hCI2.5 shown in Tab.~\ref{tab:2}).
The large errors, even at the state-specific route, combined with the formal $\order*{e^N}$ scaling of these methods,
render them unsuitable for calculations of accurate excitation energies.


%%%%%%%%%%%%%%%%%%%%%%%%%%%%%%%%
\section{Conclusion}
\label{sec:ccl}
%%%%%%%%%%%%%%%%%%%%%%%%%%%%%%%%

We conclude that 
Meanwhile, 
, and therefore not useful.


%%%%%%%%%%%%%%%%%%%%%%%%%%%%%%%%
\begin{acknowledgements}
This work was performed using HPC resources from CALMIP (Toulouse) under allocation 2021-18005.
This project has received funding from the European Research Council (ERC) under the European Union's Horizon 2020 research and innovation programme (Grant agreement No.~863481).
\end{acknowledgements}
%%%%%%%%%%%%%%%%%%%%%%%%%%%%%%%%

%%%%%%%%%%%%%%%%%%%%%%%%%%%%%%%%%%
\section*{Supporting information available}
\label{sec:SI}
%%%%%%%%%%%%%%%%%%%%%%%%%%%%%%%%%%

%%%%%%%%%%%%%%%%%%%%%%%%%%%%%%%%
%\section*{Data availability statement}
%%%%%%%%%%%%%%%%%%%%%%%%%%%%%%%%
%The data that support the findings of this study are openly available in Zenodo at \href{http://doi.org/XX.XXXX/zenodo.XXXXXXX}{http://doi.org/XX.XXXX/zenodo.XXXXXXX}.

%%%%%%%%%%%%%%%%%%%%%%%%%%%%%%%%
\bibliography{manuscript}
%%%%%%%%%%%%%%%%%%%%%%%%%%%%%%%%

\end{document}
